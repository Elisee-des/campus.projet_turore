\chapter*{INTRODUCTION GÉNÉRALE}
\markboth{\MakeUppercase{INTRODUCTION GÉNÉRALE}}{}
%\addstarredchapter{INTRODUCTION GÉNÉRALES}
\addcontentsline{toc}{chapter}{INTRODUCTION GÉNÉRALE}
\adjustmtc
\thispagestyle{MyStyle}


L'analyse sémantique des images est un domaine en pleine expansion qui trouve des applications dans divers secteurs, y compris l'agriculture et la sécurité alimentaire. Parmi les cultures vitales pour la sécurité alimentaire en Afrique subsaharienne, le manioc occupe une place prépondérante en raison de sa capacité à pousser dans des conditions environnementales difficiles. Cependant, cette culture est souvent menacée par diverses maladies qui peuvent entraîner des pertes de récoltes importantes et mettre en péril la sécurité alimentaire de millions de personnes.\par

Dans ce contexte, l'analyse sémantique des images sur les maladies du manioc émerge comme un domaine de recherche crucial. Cette approche combine les avancées en vision par ordinateur et en intelligence artificielle pour détecter, identifier et caractériser les maladies affectant les plants de manioc à partir d'images numériques. En exploitant les caractéristiques visuelles distinctives des maladies, telles que les lésions foliaires, les symptômes de pourriture des racines et les déformations des feuilles, cette analyse permet aux agriculteurs et aux chercheurs de diagnostiquer rapidement et précisément les problèmes de santé des cultures.\par

Cette introduction générale explorera les défis posés par les maladies du manioc, les applications potentielles de l'analyse sémantique des images dans ce domaine, ainsi que les avancées récentes et les perspectives d'avenir. En fin de compte, cette recherche vise à contribuer à la lutte contre les maladies du manioc, à renforcer la sécurité alimentaire et à soutenir le développement agricole durable en Afrique et au-delà.\par