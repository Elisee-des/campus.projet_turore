\documentclass{article}
\usepackage[margin=1in]{geometry} % Définit la marge à 1.5 pouces
\usepackage{amsmath}

\title{Revu littéraire de \\ Vers un modèle sémantique pour la gestion de l´interopérabilité dans le domaine de l´agriculture}
\author{Auteur : Moualkia Ammar}

\begin{document}
	\maketitle
	
	\tableofcontents
	\newpage
	
	\section{Introduction}
	La mangue joue un rôle crucial dans l'agriculture au Burkina Faso, représentant une part significative des exportations. Cependant, la détection manuelle de la maturité des mangues est sujette à des erreurs et peut entraîner des retards dans la distribution. Cette étude vise à développer un programme de détection automatisée de mangues mûres en utilisant l'algorithme YOLOv5, offrant ainsi une solution objective et efficace. L'objectif est d'améliorer les processus de tri, de récolte et de distribution, réduisant ainsi les pertes post-récolte et optimisant les opérations logistiques. L'étude aborde l'analyse, la conception et l'implémentation du programme, avec une évaluation des performances basée sur des métriques telles que la précision et le rappel. En contribuant à l'amélioration des techniques de détection, cette recherche promet d'optimiser la production et la récolte de mangues au Burkina Faso.
	
	
	\section{État de l'art}
	
	
\end{document}
